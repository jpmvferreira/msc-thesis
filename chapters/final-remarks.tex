\chapter{Final Remarks}
\label{chap:final-remarks}

In this dissertation, we constructed and used \gls{SS} mock catalogs to evaluate whether future \gls{GW} observations will be able to distinguish two different cosmological models based on non-metricity, in the form of $f(Q)$ gravity, from \gls{LCDM}. The mock catalogs were developed for the \gls{LIGO}-Virgo collaboration, as well as for future ground and space based observatories, \gls{LISA} and the \gls{ET}. The mock catalogs were generated assuming a \gls{LCDM} fiducial model.

The first model studied is the most general $f(Q)$ function that replicates a \gls{LCDM} background. It introduces only one additional free parameter with respect to \gls{LCDM}, $\alpha$, and gives rise to an effective gravitational constant such that it modifies the luminosity distance for \glspl{GW}. Due to a degeneracy between $\Omega_m$ and $\alpha$, we added \gls{SNIa} data of the Pantheon sample to fix the value of $\Omega_m$. The quality of the constrains for this model seem to increase when \gls{LISA} is able to measure events at low redshifts. We showed that the \gls{ET} is expected to be able to set the bounds on the value of $\alpha$ within a 1$\sigma$ region of $\sigma_\alpha = 0.25$. The best case scenario for \gls{LISA} shows a 1$\sigma$ region for $\alpha$ which is 0.5 larger when compared to the one expected for the \gls{ET}. The median \gls{LISA} catalog is expected to provide a 1$\sigma$ region for $\alpha$ which is 2 times larger than that of the \gls{ET}, while the worst case scenario for \gls{LISA} is expected to be almost 7 times larger than the \gls{ET}. While \gls{LIGO}-Virgo will not be able to constrain this model by itself, it will be able to improve the quality of the constrains set by \gls{LISA}, which in a best case scenario is expected to have a 1$\sigma$ region for $\alpha$ which is 0.8 larger when compared to the \gls{ET}.

The second model attempts to replace dark energy with a modification of gravity, without introducing additional free parameters, as the parameter which was inserted in the action, denoted $\lambda$, can be expressed as a function of the energy-matter content of the universe. By performing a dynamical system analysis, we showed that this model can either undergo permanent expansion, a big crunch or a \gls{CDM} type of universe depending on the sign and value of $\lambda$. To match current observations, a value of $\lambda > 0$ is chosen such that the universe expands at late times. We then verified that no number of \gls{SS} events are able to distinguish between this model and \gls{LCDM}. The same is true when using \gls{SNIa} coming from the Pantheon sample. However, it was shown that in this model, there is a tension between \gls{SNIa} and \gls{SS}, meaning that, in the future, the data obtained will be able to rule this model out.

Based on this work it is possible to see that \gls{SS} events are expected to be of great use for observational cosmology. This follows from the fact that the luminosity distance of these events is highly sensitive to modifications of gravity, both from the background and from perturbative effects. Additionally, these events are expected to be measured in a range of redshifts still to be explored. With the remarkable feature of not requiring a ladder distance in order to calibrate the measurements, they bring a new and independent way of testing our cosmological models. Additionally, the different \glspl{GW} observatories can complement each other by creating a large array of \glspl{GW} detectors, capturing events taking place throughout the universe.

In the coming future, we aim to constrain the model which was studied in \cref{chap:STG-dark-energy} with current data using various observables to ensure its consistency. In particular, we wish to see how this model behaves in the early universe by constraining it with data coming from the \gls{CMB}. Constraining this model with real data will allow us to make predictions regarding the distribution of \gls{SS} events and evaluate its differences with respect to \gls{LCDM}. Once the real data becomes available, we expect to see which one better fits to the measured events (which can be done, for instance, using the posterior predictive distribution). Additionally, we would like to generate a larger number of \gls{SS} mock catalogs in order to better understand the range of possible catalogs, and to be able to quantify how likely the more extreme scenarios are, as well as which are the most likely outcomes. We also seek to understand and implement better statistical tools in order to quantify the tension that exists between the \gls{SNIa} and \gls{SS} in the model presented in \cref{chap:STG-dark-energy}.
