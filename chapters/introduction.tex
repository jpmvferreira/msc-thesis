%---------------------------------------------------------------------------------------------------
\chapter{Introduction}
\label{chap:introduction}
%---------------------------------------------------------------------------------------------------

In this chapter, we will give a broad overview of the topics which are at the heart of this dissertation, aiming to give the reader a brief introduction on the subjects to come, which will also serve as motivation for the work carried out. In the two final sections of this chapter, we will outline the main goals for this dissertation and the structure of the document.


%---------------------------------------------------------------------------------------------------
\section{Cosmology: The Study of the Cosmos}
\label{sec:cosmology}
%---------------------------------------------------------------------------------------------------

Cosmology is the branch of physics concerned with the study of the Universe at the largest scales. Cosmology knows no physical bounds, neither in space, nor in time, taking the challenge of modeling the evolution of the Universe throughout time.

To encompass the evolution of the Universe as a whole is no easy feat, given that throughout history there were no lack of attempts. Perhaps one of the most interesting of such attempts is the Aristotelian-Ptolemaic cosmological model, one of the earliest cosmological theories that we have record of. With roots in ancient Greece, this paradigm stated that the Universe is made of spherical shells consisting of the planets and stars, as well as a handful of very primitive elements (earth, water, fire and air). Although we regard this today as an insufficient approximation of reality, worthy of a mythological tale rather than a scientific theory, it is considered until this day to be the most long-lived scientific paradigm in history \cite{MGCosmology2021}.

We might ask ourselves, how was such a naive model able to withstand for so long as a good description of the cosmos? After all, the Universe that we know exists out there differs wildly from these ideas. The reason as to why this model lived for so long was simply because the tools required to see what the Universe really look like were only developed in the centuries to come. Today, with an unprecedented amount of high quality data available to us, we have to be ever more sophisticated in order to accurately predict observations across a wide range of phenomena. Given the importance that new measurements have on the development of our ideas, and the finite number of resources we have available, choosing our investments carefully is a requirement in order to expand our knowledge as fast at we can.


%---------------------------------------------------------------------------------------------------
\section{The Standard Cosmological Model}
\label{sec:the-standard-cosmological-model}
%---------------------------------------------------------------------------------------------------


%%% ΛCDM intro
In our attempts to contain the evolution of the cosmos in a single theory, the standard model of cosmology, referred to as \gls{LCDM}, was developed. Starting its roots in the 20th century, it is till today the dominant paradigm in modern cosmology and while remaining fairly simple in its theoretical formulation, is able to successfully account for most of the observed phenomena.

%%% GR intro
One of the requirements in order to understand the evolution of the cosmos, is the ability to describe the motion of bodies when acted upon by gravity. Although there are several theories of gravity which one can pick from, Einstein's theory of \gls{GR} is undoubtedly the most successful one so far. Since the formulation of \gls{GR}, we see the motion of a particle in a gravitational field as the consequence of the curvature of spacetime, a four dimensional manifold where space and time are fused together into one mathematical object. Thus, knowing how spacetime curves, which can be computed knowing just how much matter content there is, means that one can make predictions regarding the future state of any given particle. Initially starting with the prediction of the perihelion precession of Mercury and the deflection of light rays, when the Universe was thought to be our galaxy and a background of stars, it successfully accounts for a wide range of phenomena which extended far beyond the known Universe at that time. For all of its predictions and agreement with experiment, this theory is, therefore, at the heart of the standard cosmological model.

%%% matter energy content
While during the early years of \gls{GR} there were few observations of the Universe at cosmological scales, this rapidly changed due to the major technological developments that followed. Perhaps a few of the more surprising observations were the detection of the excessive speed of the galaxies and missing mass observed in the Coma galaxy cluster, by Zwicky in 1933 \cite{Zwicky1933}, and in 1998 where measurements from \gls{SNIa} were used to show an accelerating expanding Universe \cite{SupernovaCosmologyProject1998}. These observations lead to the concept of dark matter and dark energy respectively, which are known today to be a major portion of our Universe. The other part of our Universe consists of less exotic components, which surround us in our everyday life, which we differentiate between ordinary matter and radiation.

%%% cosmological principle
Given that in \gls{GR} one requires the knowledge of the matter distribution in order to compute the geometry of the Universe, we are lead, by observations, to the idea that at sufficiently large scales, the Universe is homogeneous (i.e. no point is privileged when compared to the other) and isotropic (i.e. at a given point, the Universe looks the same in all directions). This is known as the cosmological principle, and is one of the main ideas behind the standard cosmological model. Recent observations have lead the community to favor a universe which is spatially flat.

%%% achievements of ΛCDM
With this surprisingly straightforward theoretical formulation, this model as been able to accurately account for many of the observed phenomena at largest scale, of which the most prominent are \cite{Amendola2010}:

\begin{enumerate}
    \item The existence of the \gls{CMB};
    \item The accelerated expansion of the Universe;
    \item The abundances of the light elements;
    \item The large scale distribution of galaxies.
\end{enumerate}

%%% problems with ΛCDM
However, with respect to those same observations, \gls{LCDM} is currently facing some observational challenges. As listed in \cite{LCDM-R2021}, the most significant are:

\begin{enumerate}[font=\bfseries]
    \item \textbf{The Hubble crisis}: A tension between the measurements of the Hubble constant between low and high redshift observations;
    \item \textbf{Anomalies in the anisotropies of \gls{CMB}}: Unexplained distribution of the temperature fluctuations in \gls{CMB} which are not accounted for in \gls{LCDM};
    \item \textbf{The $\boldsymbol{\sigma}_{\boldsymbol{8}}$ tension}: A tension on the amplitude of the matter power spectrum at the scale $8 h^{−1}$ Mpc between large scale structure data and Plack measurements.
\end{enumerate}

\noindent Additionally, one could also list other problems, such as the lack of theoretical explanation of the two most prevalent components of our Universe, dark energy and dark matter. The standard model of particle physics fails to predict the energy density value of the dark energy by a factor of $10^{120}$ \cite{Carroll2001}. Also the incompatibility of \gls{GR} with well established microscopic theories, which was the driving source of evolution for several modifications to gravity in starting the past century.


%---------------------------------------------------------------------------------------------------
\section{Beyond the Standard Model}
\label{sec:alternative-cosmological-models}
%---------------------------------------------------------------------------------------------------

%%% seeking new models
Faced with the issues we have just discussed in the previous section, one should seek new models that might improve on \gls{LCDM}. There is a countless number of ways to modify \gls{LCDM}, in this work, we have decided to explore possible modifications to \gls{GR}, the theory of gravity at the core of \gls{LCDM}.

%%% gravity as geometry
Einstein initially formulated \gls{GR} as a geometrical theory of gravity where the gravitational phenomena are described by the curvature of spacetime. Despite being the more common approach, the geometry of spacetime might not necessarily be due to curvature but other geometrical objects, such as torsion and non-metricity. These alternative descriptions of gravity, that preserves the equivalence principle, can be used to create theories that may or may not resemble \gls{GR}. The theories equivalent to \gls{GR}, often referred to as different interpretations of \gls{GR}, have been recently popularized in what is now known as the geometrical trinity of gravity \cite{Jimenez2019}.

%%% first attempts at using torsion and non-metricity
The idea of using other geometrical objects in an attempt to explain for unaccounted phenomena by creating modified theories of gravity is by no means contemporary. One of the earliest attempts at modifying gravity was developed by Weyl, in 1919, where he used non-metricity to attempt to unify gravity and electromagnetism \cite{Bergmann1976}. Later, in 1923, Cartan developed a modified theory of gravity based on torsion instead \cite{Tonnelat2014}.


In this work we will pursue the path of modifying gravity and will rely on a non-metricity scalar, $Q$, by considering an action that depends on an arbitrary function of this quantity, $f(Q)$. We will then see whether specific forms of $f(Q)$ are able to compete against the standard cosmological model.


%---------------------------------------------------------------------------------------------------
\section{Standard Sirens}
\label{sec:standard-sirens}
%---------------------------------------------------------------------------------------------------

%%% GW astronomy
Much like we have hinted previously during this introduction, data is essential to bring any physical theory onto solid ground. In order to further test our models for both gravity and cosmology, we are now beginning an era which brings a new way to study the Universe, \gls{GW} astronomy. This new way to ``listen'' to the Universe, has already provided us with remarkable results, and further are to be expected in the future from future ground and space based observatories which are currently being developed. With this new source of data, we will be able to further address the difficulties that \gls{LCDM} faces, and also to study gravity in the high energy regime, allowing us to investigate deviations from \gls{GR}.

%%% what are SS
A specific type of phenomena of particular interest to cosmology are \gls{SS} events. They are characterized by the emission of both \glspl{GW} and \gls{EM} radiation, which are expected to take place in the merger of binary system of compact objects, due to the decrease of the orbital radius by slowly loosing energy from the gradual emission of \glspl{GW}.

%%% why are SS interesting for cosmology
The reason why these events are of such relevance for cosmology is the possibility of obtaining the value of the luminosity distance for that event directly from the \gls{GW}, and, from the \gls{EM} counterpart, its corresponding redshift. With both these measurements it is possible to reconstruct the Hubble diagram, directly constraining the parameters for the model being considered, without relying on a distance ladder.

%%% lack of SS events means we have to resort to mock catalogs
Although the detection of such events is no shorter than a remarkable achievement, so far there has only been one confirmed \gls{SS} event, named GW170817 \cite{GW170817}, and a proposed \gls{EM} counterpart to GW190521 \cite{GW190521} was proposed in \cite{GW190521-EM}. Although this first event was measured to incredible accuracy, and showed to be in agreement with \gls{LCDM}, it is not nearly enough to constrain the models we will consider throughout this dissertation. Having in mind forthcoming data from current and future \gls{GW} observatories, we will develop \gls{SS} mock catalogs to investigate whether we will be able to distinguish our models from \gls{LCDM}.


%---------------------------------------------------------------------------------------------------
\section{Objectives}
\label{sec:objectives}
%---------------------------------------------------------------------------------------------------

In short, the main goals of this dissertation are to:
\begin{enumerate}[font=\bfseries]
    \item Generate realistic mock catalogs of \gls{SS} events for future and current \gls{GW} observatories;
    \item Understand how cosmological models based on a specific form on non-metricity modify the propagation of \glspl{GW};
    \item Study two specific cosmological models based on non-metric gravity;
    \item See whether future \gls{SS} events will be able to distinguish between these models and \gls{LCDM}.
\end{enumerate}
Although our analysis is being developed for two specific modified gravity cosmological models, this procedure allows us to forecast the constraints set by \gls{SS} events for any provided cosmological model.


%---------------------------------------------------------------------------------------------------
\section{Structure of this Document}
\label{sec:structure-of-this-document}
%---------------------------------------------------------------------------------------------------

The first three chapters of this dissertation are used to explain the necessary background required to understand the two following chapters, which are the ones that feature original work, while the last chapter includes some final remarks. The outline of each chapter is as follows:
\begin{itemize}
    \item \Cref{chap:gravity} will introduce gravity as a geometrical theory, explore some of its objects, and lay the foundations for the non-metric theories of gravity we will consider;
    \item \Cref{chap:cosmology} will detail the assumptions made regarding the Universe at the cosmological scales, and develop the relevant portions of both \gls{LCDM} and cosmology based on $f(Q)$;
    \item \Cref{chap:datasets} introduces the datasets that we have used throughout this dissertation, the procedure employed to generate the \gls{SS} mock catalogs and the methodology for the Bayesian analysis, for both the model and catalog selection criteria;
    \item \Cref{chap:STG-LCDM-bg} introduces a model of $f(Q)$ gravity that features a \gls{LCDM} background, which we then forecast using \gls{SS} events. This chapter led to original work which was published in \cite{Ferreira2022};
    \item \Cref{chap:STG-dark-energy} introduces an $f(Q)$ model with the aim of replacing dark energy. In order to look for viable cosmologies we perform a dynamical system analysis and apply model selection criteria using both \gls{SS} and \gls{SNIa} events;
    \item \Cref{chap:final-remarks} includes an outline of this dissertation and presents some foreseeable future work.
\end{itemize}
