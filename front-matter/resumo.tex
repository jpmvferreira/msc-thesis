\selectlanguage{portuguese}

\vspace*{2cm}
\begin{center}
    \Large \bf Resumo
\end{center}
\vspace*{1cm} \setlength{\baselineskip}{0.6cm}

%%% overview
A nossa dissertação tem como objetivo estudar dois modelos cosmológicos, que têm como base teorias de gravidade modificadas que envolvem à não-metricidade, em vez da tradicional teoria da Relatividade Geral. Recorrendo a catálogos simulados de sirenes padrão, vamos prever a capacidade de futuros observatórios de ondas gravitacionais conseguirem constranger os parâmetros presentes nesses mesmos modelos, bem como se serão capazes de os distinguir do modelo padrão da cosmologia.

%%% ΛCDM
O modelo padrão da cosmologia, conhecido como \gls{LCDM}, foi desenvolvido ao longo do último século com o intuito de descrever o Universo na sua maior escala. Sendo relativamente simples na sua formulação teórica, tendo por base um universo isotrópico e homogéneo com interações descritas pela teoria da Relatividade Geral, tem sido validado nas últimas duas décadas através das suas previsões, face a observações feitas com cada vez maior precisão. No entanto, apesar dos seus grandes sucessos, existem ainda várias questões por esclarecer. Em particular, a existência de observações em tensão entre si, das quais se destaca a tensão de Hubble, a anomalia na radiação cósmica de fundo, bem como a falta de deteção em laboratório dos dois componentes mais prevalecentes do universo, a matéria e energia escura.

%%% non-metricity
Existem, naturalmente, inúmeras abordagens possíveis para tentar encontrar um novo modelo que seja capaz de resolver os problemas que existem em torno de \gls{LCDM}. Nesta dissertação, vamos considerar teorias de gravidade modificada baseadas em não-metricidade, que foram inicialmente propostas com o intuito de unificar a gravitação com o eletromagnetismo, e vão ser agora utilizadas para construir modelos cosmológicos alternativos ao modelo padrão da cosmologia.

%%% standard sirens
A Relatividade Geral como a descrição da interação gravitacional teve como consequência a previsão de vários fenómenos até então inobserváveis. Um desses fenómenos, que tem particular interesse para esta dissertação, é a existência de ondas gravitacionais, que foram observadas pela primeira vez por uma análise indireta do sistema binário de Hulse-Taylor em 1974, e mais tarde, em 2015, medidas diretamente pelo \gls{LIGO}. Estes eventos têm como principal origem sistemas binários de corpos extremamente compactos (por exemplo buracos negros, estrelas de neutrões). Estes eventos são possíveis pois na teoria da Relatividade Geral os sistemas binários são instáveis, emitindo ondas gravitacionais gradualmente, fazendo com que a distância entre os corpos diminua, até
à sua colisão. No caso em que os corpos compactos emitem também ondas eletromagnéticas, como por exemplo no caso da colisão de duas estrelas de neutrões, temos um evento que se chama sirene padrão. Estes eventos são de extrema importância para o estudo da cosmologia, pois é possível obter o desvio para o vermelho a partir da onda eletromagnética e a distância luminosa a partir da onda gravitacional. Isto permite-nos constranger um modelo cosmológico sem recorrer a uma escada de distâncias cosmológicas, evitando assim eventuais erros de calibração.

%%% mock catalogs
Até à data apenas houve uma ocorrência confirmada de um evento deste tipo, o GW170817. Não é portanto possível constranger modelos com dados actuais de sirenes padrão, e vamos alternativamente, criar catálogos simulados de sirenes padrão, com o intuito de conseguir saber se no futuro será possível distinguir os modelos cosmológicos que vamos considerar ao longo desta dissertação do modelo padrão da cosmologia.

%%% 1º chapter
No primeiro capítulo desta dissertação vamos fazer uma introdução geral aos conceitos que vamos desenvolver ao longo do documento e que vão também servir de motivação para o trabalho desenvolvido. Iremos destacar qual o objetivo principal da cosmologia como área de estudo, algumas das características e falhas de \gls{LCDM}, o facto de ser possível ver a gravidade como uma consequência da geometria do espaço-tempo e uma breve introdução a sirenes padrão. Os principais objetivos do trabalho, bem como a estrutura do documento, também estarão presentes nesse mesmo capítulo.

%%% 2º chapter
No segundo capítulo, iremos apresentar a gravidade como a consequência da geometria do espaço-tempo. Vamos também mostrar que, ao contrário do que acontece na teoria da Relatividade Geral, não é necessária a existência de curvatura para descrever esta interação. A fim de poder fazer essa construção, vamos introduzir brevemente o conceito de não-metricidade e de torção, dois objetos que são assumidos ser nulos na teoria da Relatividade Geral. Com estes objetos vamos mostrar que é possível construir três teorias equivalentes entre si, na qual uma delas é uma teoria exclusivamente não métrica, baseada no escalar da não-metricidade $Q$, a outra é baseada apenas em torção, e a terceira é a teoria da Relatividade Geral, que depende unicamente da curvatura. De seguida vamos generalizar cada uma destas teorias promovendo a quantidade escalar na ação para uma função arbitrária do mesmo, com particular interesse na teoria de $f(Q)$, que estará no centro desta dissertação.

%%% 3º chapter
No terceiro capítulo, iremos introduzir formalmente as considerações por detrás dos nossos modelos cosmológicos. Ao longo desta dissertação foi considerado que o Universo nas grandes escalas é homogéneo e isotrópico, bem como espacialmente plano, e encontra-se preenchido por um fluido perfeito. Em seguida, vamos apresentar o modelo padrão da cosmologia e as modificações que têm lugar ao considerar modelos cosmológicos baseados em $f(Q)$, tanto na evolução do fundo cósmico como na propagação de ondas gravitacionais.

%%% 4º chapter
No quarto capítulo vamos expor as fontes de dados que utilizámos ao longo do trabalho, desde as Supernova tipo Ia ao processo de gerar catálogos simulados de sirenes padrão para todos os observatórios considerados. Para isso, vamos descrever as características dos detetores de ondas gravitacionais da colaboração \gls{LIGO}-Virgo, bem como as expectativas existentes para futuros observatórios como o \gls{LISA} e o \gls{ET}. O final deste capítulo será dedicado à metodologia da nossa análise estatística, descrevendo como e com que ferramentas foram realizados os constrangimentos dos modelos, bem como o processo de seleção de modelos e de catálogos.

%%% 5º chapter
O quinto capítulo será dedicado ao modelo de $f(Q)$ mais geral que replica a dinâmica de fundo de \gls{LCDM}, apresentado diferenças somente na propagação de perturbações. Ao introduzir apenas um parâmetro adicional quando comparado com \gls{LCDM}, denominado por $\alpha$, iremos analisar qual é a capacidade de cada observatório de colocar limites no valor do parâmetro $\alpha$. Foi observado que o \gls{LIGO}-Virgo não é capaz de colocar limites ao valor de $\alpha$, enquanto que o \gls{ET} e o \gls{LISA} são, com o \gls{ET} a colocar limites mais fortes do que o \gls{LISA}. Foi observado que tanto o \gls{LIGO}-Virgo como o \gls{LISA} sofrem de flutuações estatísticas nos seus catálogos, pelo que serão considerados três casos representativos: o melhor, o mediano e o pior. Pelo contrário, o \gls{ET} não sofre destas flutuações estatísticas, dado que o número de eventos é suficientemente grande para que essas flutuações sejam desprezáveis. Mostrámos também que mesmo se no futuro se obtiver um mau catálogo do \gls{LISA}, podemos utilizar os dados do \gls{LIGO}-Virgo para aproximar esse catálogo a um catálogo mediano do \gls{LISA}.

%%% 6º chapter
No sexto capítulo iremos estudar um modelo de $f(Q)$ que tem como objetivo substituir a existência de energia escura através da introdução de uma forma específica na modificação à gravidade. Em primeira instância foi feita uma análise recorrendo a técnicas de sistemas dinâmicos para analisar regiões no espaço de parâmetros que apresentam cosmologias que estão de acordo com as observações atuais. Utilizando o processo de seleção de modelos introduzido previamente, recorrendo apenas a sirenes padrão, é impossível distinguir este modelo de $\Lambda$CDM. No entanto, ao introduzirmos os dados de Supernovas do tipo Ia, surgem tensões no modelo quando comparado com sirenes padrão. Significa isto que, no futuro, será possível utilizar dados de sirenes padrão para distinguir este modelo de \gls{LCDM}.

%%% 7º chapter
No sétimo e último capítulo encerramos a dissertação fazendo considerações finais, bem como apresentar perspetivas futuras para dar continuidade ao trabalho aqui desenvolvido.

\vfill
\begin{flushleft}
    \textbf{Palavras-chave:}
    Sirenes Padrão, Ondas Gravitacionais, Astronomia Multimensageira, Gravidade Modificada, Cosmologia Observacional
\end{flushleft}

\selectlanguage{english}
