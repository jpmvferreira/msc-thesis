% > maximum of 300 words

\vspace*{2cm}
\begin{center}
    \Large \bf Abstract
\end{center}
\vspace*{1cm} 
\setlength{\baselineskip}{0.6cm}

% don't expand the following acronyms
\glsunset{LIGO}
\glsunset{LISA}
\glsunset{ET}
\glsunset{STG}
\glsunset{GW}
\glsunset{SNIa}

%%% main goal
In this dissertation, we study two cosmological models based on $f(Q)$ gravity. We resort to mock catalogs of \gls{SS} events to see whether data from future \gls{GW} observatories will be able to distinguish these models from \gls{LCDM}.

%%% model nº1
The first model is the most general $f(Q)$ formulation that replicates a \gls{LCDM} background, with deviations appearing only at the perturbative level. It has one additional free parameter compared to \gls{LCDM}, $\alpha$, which when set to zero falls back to \gls{LCDM}. We show that \gls{LIGO}-Virgo is unable to constrain $\alpha$, due to the high error and low redshift of the measurements, whereas \gls{LISA} and the \gls{ET} will, with the \gls{ET} outperforming \gls{LISA}. The catalogs for both \gls{LISA} and \gls{LIGO}-Virgo show non-negligible statistical fluctuations, where we consider three representative catalogs (the best, median and worst), whereas for the \gls{ET}, only a single catalog is considered, as the number of events is large enough for statistical fluctuations to be neglected. The best \gls{LISA} catalog is the one with more low redshift events, while the worst \gls{LISA} catalog features fewer low redshift events. Additionally, if we are to observe a bad \gls{LISA} catalog, we can rely on data from \gls{LIGO}-Virgo to improve the quality of the constrains, bringing it closer to a median \gls{LISA} catalog.

%%% model nº2
The second model attempts to replace dark energy by making use of a specific form of the function $f(Q)$. We study this model resorting to dynamical system techniques to show the regions in parameter space with viable cosmologies. Using model selection criteria, we show that no number of \gls{SS} events is, by itself, able to tell this model and \gls{LCDM} apart. We then show that if we add current \gls{SNIa} data, tensions in this model arise when compared to the constrains set by the \gls{SS} events.

\vfill
\begin{flushleft}
    \textbf{Keywords:}
    Multi-messeger Astronomy, Modified Gravity, Observational Cosmology, Standard Sirens, Gravitational Waves
\end{flushleft}
